Sea \(c_n\) el n\'umero de formas de teselar una cinta circular de tama\(\tilde{n}\)o \(1 \times 1\) o \(1 \times 2\). Por ejemplo, a partir de la siguiente figura se puede verificar que \(c_4 = 7\):
\shorthandoff{>}
\begin{tikzpicture}[scale=1.2,cap=round,line join=round]

  % A helper macro to draw a "washer" (ring) of 5 arcs at position (#2,#3),
  % skipping radial boundaries whose angles are listed in #1.
  % #1 is a comma-separated list of angles, e.g. {72,216} means
  % "do *not* draw the boundary lines at 72° and 216°".
  %
  % For each ring:
  %   -- Outer radius = 1
  %   -- Inner radius = 0.7
  %
  \newcommand\drawRing[3]{
    % #1 = set of skip angles, comma-separated
    % #2,#3 = shift in x,y
    \begin{scope}[shift={(#2,#3)}]
      % Draw the outer green ring and inner white circle
      \draw[thick,fill=green!30] (0,0) circle (1);
      \draw[thick,fill=white] (0,0) circle (0.7);
  
      % For each of the 5 radial lines (0°,72°,144°,216°,288°),
      % check if it is in the skip list. If not, draw it.
      \foreach \a in {0,72,144,216,288} {
        % We'll see if \a matches one of the angles in #1
        \pgfkeys{/skipfound=false}%
        \foreach \skipA in #1 {
          \ifx\skipA\empty
            % If #1 is empty, do nothing
          \else
            \ifnum\skipA=\a
              \pgfkeys{/skipfound=true}%
            \fi
          \fi
        }
        % If we never found \a in the skip list, draw the line
        \pgfkeysifnum{\pgfkeysvalueof{/skipfound}}=0{
           \draw[thick] (\a:0.7) -- (\a:1);
        }{}
      }
    \end{scope}
  }
  
  % The 11 valid skip patterns for c5:
  %
  % Interpreting a 5-bit circular pattern:
  %  - '1' at position i => skip boundary at angle 72*i.
  %  - No two '1's are adjacent in a circular sense.
  % This yields exactly the following 11 sets:
  %
  % 1)  00000 -> skip none
  % 2)  10000 -> skip angle 0
  % 3)  01000 -> skip angle 72
  % 4)  00100 -> skip angle 144
  % 5)  00010 -> skip angle 216
  % 6)  00001 -> skip angle 288
  % 7)  10100 -> skip angles 0,144
  % 8)  01010 -> skip angles 72,216
  % 9)  00101 -> skip angles 144,288
  % 10) 10010 -> skip angles 0,216
  % 11) 01001 -> skip angles 72,288
  %
  % We will place them in a 4-column grid:
  
  % (Row 0)
  \drawRing{}             {0}{0}      % no skips
  \drawRing{{0}}          {3}{0}      % skip 0
  \drawRing{{72}}         {6}{0}
  \drawRing{{144}}        {9}{0}
  % (Row 1)
  \drawRing{{216}}        {0}{-3}
  \drawRing{{288}}        {3}{-3}
  \drawRing{{0,144}}      {6}{-3}
  \drawRing{{72,216}}     {9}{-3}
  % (Row 2)
  \drawRing{{144,288}}    {0}{-6}
  \drawRing{{0,216}}      {3}{-6}
  \drawRing{{72,288}}     {6}{-6}
  \end{tikzpicture}
  \shorthandon{>}

  % \caption{Las 7 formas distintas de teselar la cinta circular para \(n=4\). 
  % El \'octavo anillo (abajo a la derecha) est\'a en blanco.}
