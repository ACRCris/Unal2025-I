\usepackage{caption}
\usepackage[spanish]{layout}%Este paquete permite visualizar el formato de la página en español. toca colocar \layot despues de \begin{document}%
% [pdftex][dvips]
\usepackage{parskip}
%\usepackage{color,graphicx} Para poder insertar graficas%
\usepackage[lmargin=2.4cm,rmargin=1.5cm, top=4.3cm, bottom=2cm]{geometry}
\usepackage{amsmath}%Insertar ecuaciones y matematicas
\usepackage{amssymb}%Insertar unos simbolos matematicos especiales%
\usepackage{comment}%comentarios (RENZO)
\usepackage{setspace}
\usepackage{empheq}
\usepackage{multicol}
\usepackage{lipsum}
\usepackage{xspace}
\usepackage{xcolor}
\usepackage{csquotes}
\onehalfspacing
\usepackage[mathscr]{euscript}%Tipo especial de letra%
\usepackage[utf8]{inputenc}%Para las tildes
\usepackage[spanish,activeacute]{babel}%Todo en Español
%\usepackage{multicol}
\pagestyle{empty}
%\usepackage{spalign}  %SISTEMAS DE ECUACIONES%
%\usepackage{array}
\usepackage{layout}
\usepackage{picinpar}
\setlength{\footskip}{1cm}
\usepackage{color, graphicx}
\usepackage{tikz}
\usetikzlibrary{positioning, automata}
\usepackage{fancyhdr}
\usepackage{bm}
\usepackage{extramarks}
\usepackage[plain]{algorithm}
\usepackage{algpseudocode}
\usepackage[T1]{fontenc}    
\usepackage{url}        
\usepackage{booktabs}
\usepackage{pdfpages}
\usepackage{amsfonts} 
\newtheorem{lema}{Lema}
\usepackage{nicefrac}       
\usepackage{microtype}   
\usepackage{multicol}
\usepackage[spanish]{babel}
\usepackage{amsmath}
\usepackage{graphicx}
\usepackage{float}
\usepackage{caption}
\usepackage{ragged2e}
\usepackage{wrapfig}
\usepackage{mathtools}
\usepackage[mathscr]{euscript}
\let\euscr\mathscr\let\mathscr\relax% just so we can load this and rsfs
\usepackage[scr]{rsfso}
\newcommand{\powerset}{\raisebox{.15\baselineskip}{\Large\ensuremath{\wp}}}
%\usepackage{geometry}
 %\geometry{
 %a4paper,
 %total={170mm,257mm},
 %left=15mm,
 %top=20mm,
 %}
\usepackage{comment}
\usepackage[makeroom]{cancel}

\usepackage{endnotes}
\usepackage[hyperfootnotes=false]{hyperref}

\usepackage{tikz}


\newcommand {\Z}{\mathbb{Z}}
\newcommand {\N}{\mathbb{N}}
\newcommand {\Q}{\mathbb{Q}}
\newcommand {\C}{\mathbb{C}}
\newcommand {\R}{\mathbb{R}}
\newcommand {\F}{\mathcal{F}}
\newcommand{\solution}{\newline\\ \textbf{R/ }}
%newcommand{\log}{\mbox{log }}

\newcommand{\proof}{\newline \textbf{D/ }}
\pagestyle{fancy}

\fancyhead[R]{ \textsc{IAC 2025-I}\\ \small \textit{Parcial final}} 
\fancyhead[L]{ \hspace{0.2cm} \thepage \vspace{0.1cm}}

\usepackage{tcolorbox}
\newtcolorbox{solutionbox}{
    colback=white,
    colframe=white,
    width=\linewidth,
    boxrule=0pt, % Sin borde
    arc=0pt, % Sin esquinas redondeadas
}

\usepackage{amsmath}
\usepackage{listings}
\usepackage{xcolor}
\definecolor{codegreen}{rgb}{0,0.6,0}
\definecolor{codegray}{rgb}{0.5,0.5,0.5}
\definecolor{codepurple}{rgb}{0.58,0,0.82}
\definecolor{backcolour}{rgb}{0.95,0.95,0.92}
\usepackage{amssymb,amsthm,amsfonts,amscd,mathrsfs}
\usepackage{booktabs}
\usepackage{graphicx} % Required for inserting images
\lstdefinestyle{mystyle}{
    backgroundcolor=\color{backcolour},   
    commentstyle=\color{codegreen},
    keywordstyle=\color{magenta},
    numberstyle=\tiny\color{codegray},
    stringstyle=\color{codepurple},
    basicstyle=\ttfamily\footnotesize,
    breakatwhitespace=false,         
    breaklines=true,                 
    captionpos=b,                    
    keepspaces=true,                 
    numbers=left,                    
    numbersep=5pt,                  
    showspaces=false,                
    showstringspaces=false,
    showtabs=false,                  
    tabsize=2
}

\lstset{style=mystyle}